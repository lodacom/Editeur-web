\documentclass[a4paper, 12pt]{report}

\title{Editeur Web}
%% Pour la police de caractères.
\usepackage{fontspec}
%% Pour la langue des titres et sous-titres.
\usepackage[francais]{babel}
%% Pour de belles images.
\usepackage{graphicx}
%% Pour des marges plus équitables.
\usepackage[margin=2cm]{geometry}
%% Pour faire le glossaire.
\usepackage[xindy]{glossaries}
\makeglossaries
%% La police de caractères.
\setmainfont{Delicious-Roman}
%% Pour pouvoir écrire du code.
%\usepackage{minted}
%\usemintedstyle{tango}
\begin{document}
	\begin{titlepage}
		\center{\includegraphics[width=5cm]{images/logoUM2.png}}	\\ 
		~\\
		~\\
		~\\
		~\\		
		\begin{center}
			{\large Rapport préliminaire de projet} \\
			{\large Licence 3}\\
			\vspace{1,5cm}
			{\Huge Editeur de sites web}\\
			~\\
			~\\
			~\\
			\includegraphics[width=12.5cm]{images/logoTest1.png}
			~\\
			~\\
			{\large Réalisé par :} \\
			~\\
			{\LARGE Pierre Burc, Olivier Duplouy, \\
				      Hamza Erraji, Issam Amal,\\
				      Mickaël Berger, Joachim Divet,\\
				      Zidane Sadiki et Abdelhamid Belarbi}\\
			\vspace{1,5cm}
			{\large Sous la direction de :} \\
			~\\
			{\LARGE Michel Meynard} \\
			\vspace{2.5cm}
			{\large Année universitaire 2011-2012}			
		\end{center}
	\end{titlepage}
%%%%%%%%%%%%%%%%%%%%%%%%%%%%%%%%%%%%%%%%%%%%%%%%%%%%%%%%%%%%%%%%%%%%%%%%%%%%%%%%%%%%%%%%%%%%%%%%%%%%%%%%%%%%%%%%%%%%%%%%%%%%%%%%%%
%%%%%%%%%%%%%%%%%%%%%%%%%%%%%%%%%%%%%%%%%%%%%%%%%%%%%%%%%%%%%%%%%%%%%%%%%%%%%%%%%%%%%%%%%%%%%%%%%%%%%%%%%%%%%%%%%%%%%%%%%%%%%%%%%%
\begin{chapter}*{Avant-propos}
	Il serait profondément malsain de commencer la description de ce projet sans, avant celà, une touche de présentation.\\
	Nous sommes donc huit étudiants de la Faculté des sciences de Montpellier, dans le domaine de l'informatique.
	Nous fûmes réunis pour la première fois en janvier 2012 pour débattre sur le sujet de notre projet, à savoir:\\ 
	\quotation{\emph{"Dans le cadre du développement de sites web, on souhaiterait utiliser un éditeur multi-fichiers permettant de réaliser différentes actions sur des fichiers relatifs à un site."}}\\


    Après cette réunion nous avons à force de discussions, d'argumentations opposées, d'idées nouvelles et de feuilles de papier décidés d'une 
    marche à suivre pour répondre à cette problématique.\\
    Et c'est ainsi que dans les pages qui vont suivre sont exposées nos conclusion quant à la mise en place d'une solution.	
\end{chapter}

	\begin{part}{Conception}
		\begin{chapter}{Analyse de l'existant}
		\end{chapter}
		\begin{chapter}{Cahier des charges}
		\end{chapter}
		\begin{chapter}{Modélisation}
		\end{chapter}
	\end{part}
	\begin{part}{Planification}
		\begin{chapter}{Contraintes}
		\end{chapter}
		\begin{chapter}{Choix des outils}
		\end{chapter}
	\end{part}

\end{document}
